\documentclass[a4paper]{report}

\usepackage[T1]{fontenc}
\usepackage[utf8]{inputenc}
\usepackage[english]{babel}

\usepackage{booktabs}
\renewcommand{\arraystretch}{1.2}

\setlength{\textwidth}{16cm} \setlength{\textheight}{23cm}
\setlength{\oddsidemargin}{0cm} % +0.5 si {\textwidth}{15cm} ; -0,5 si {\textwidth}{15cm}
\setlength{\headheight}{0cm} \setlength{\topmargin}{0.3cm}
\setlength{\headsep}{0cm}

\author{Clément Decoodt, Alexis Bauvin, Alexandre Janniaux}
\title{PR1 SE201: execution platforms}

\begin{document}

\maketitle

\section{Two's complement arithmetic}

\subsection{}

\subsection{}

\subsection{}

\subsection{Multiplication and overflow}

\section{Processor design}

\subsection{Instruction set architecture}

We defined our architecture set instruction as the following:

\subsubsection{R-type instructions}

Instruction structure :

\begin{center}
	\begin{tabular}{|l|l|l|l|}
		\hline
		opcode & output register & input register 1 & input register 2 \\
		\hline
		7 bits & 3 bits & 3 bits & 3 bits \\
		\hline
	\end{tabular}
\end{center}

Instruction list :

\begin{center}
	\begin{tabular}{|l|l|l|}
		\hline
		Opcode & Mnemonic & Description \\
		\hline \hline
		\texttt{0000000} & \texttt{NOP} &  Does nothing \\
		\texttt{0000011} & \texttt{ADD} &  Adds two signed numbers \\
		\texttt{0000111} & \texttt{ADDU} & Adds two unsigned numbers \\
		\texttt{0000010} & \texttt{SUB} &  Substracts two signed numbers \\
		\texttt{0000110} & \texttt{SUBU} & Substracts two unsigned numbers \\
		\texttt{0000100} & \texttt{MULU} & Multiplied two unsigned numbers \\
		\hline
	\end{tabular}
\end{center}

Example:

\texttt{sub \$r1 \$r2 \$r3} : computes \texttt{\$r2} $-$ \texttt{\$r3} and stores the
result in \texttt{\$r1}. \\
Encodes to ${
	\underbrace{\texttt{0000010}}_{opcode}
	\underbrace{\texttt{001}}_{out}
	\underbrace{\texttt{010}}_{in 1}
	\underbrace{\texttt{011}}_{in 2}
}_2$ = \texttt{0x0453}$_{16}$

\subsubsection{I-type instructions}

Instruction structure :

\begin{center}
	\begin{tabular}{|l|l|l|}
		\hline
		opcode & output register & immediate value \\
		\hline
		2 bits & 3 bits & 11 bits \\
		\hline
	\end{tabular}
\end{center}

Instruction list :

\begin{center}
	\begin{tabular}{|l|l|l|}
		\hline
		Opcode & Mnemonic & Description \\
		\hline \hline
		\texttt{10} & \texttt{ADDIU} & Adds-accumulate an unsigned immediate
		                               value to output register \\
		\texttt{11} & \texttt{LWI} &   Copies the immediate value to
		                               the register \\
		\hline
	\end{tabular}
\end{center}

\subsubsection{J-type instructions}

Instruction structure :

\begin{center}
	\begin{tabular}{|l|l|l|l|}
		\hline
		opcode & register 1 & register 2 & immediate value \\
		\hline
		4 bits & 3 bits & 3 bits & 6 bits \\
		\hline
	\end{tabular}
\end{center}

Instruction list :

\begin{center}
	\begin{tabular}{|l|l|l|}
		\hline
		Opcode & Mnemonic & Description \\
		\hline \hline
		\texttt{0010} & \texttt{LW} & Loads word from memory at address
		                              \texttt{\$r2 + immediate} (offset) into \\
		              &             & first register \\
		\texttt{0011} & \texttt{SW} & Stores word in \texttt{\$r1} to memory at
		                           address \texttt{\$r2 + immediate} (offset) \\
		\hline
	\end{tabular}
\end{center}

\subsubsection{K-type instructions}

Instruction structure :

\begin{center}
	\begin{tabular}{|l|l|l|l|}
		\hline
		opcode & immediate value & register 1 & register 2 \\
		\hline
		2 bits & 8 bits & 3 bits & 3 bits \\
		\hline
	\end{tabular}
\end{center}

Instruction list :

\begin{center}
	\begin{tabular}{|l|l|l|}
		\hline
		Opcode & Mnemonic & Description \\
		\hline \hline
		\texttt{01} & \texttt{BNE} & Jumps to \texttt{pc + immediate} if
		                             \texttt{\$r1 $\neq$ \$r2} \\
		\hline
	\end{tabular}
\end{center}

\subsection{Pipelining}

\end{document}


