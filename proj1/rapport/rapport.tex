\documentclass{report}

\usepackage[T1]{fontenc}
\usepackage[utf8]{inputenc}
\usepackage[english]{babel}

\usepackage{booktabs}
\renewcommand{\arraystretch}{1.2}


\author{Clément Decoodt, Alexis Bauvin, Alexandre Janniaux}
\title{PR1 SE201: execution platforms}

\begin{document}

\maketitle

\section{Two's complement arithmetic}

\subsection{}

\subsection{}

\subsection{}

\subsection{Multiplication and overflow}

\section{Processor design}

\subsection{Instruction set architecture}

We defined our architecture set instruction as the following:

\subsubsection{Memory instructions}

\begin{tabular}{@{}ll@{}}
    LW          & Load word from memory to register                      \\
    SW          & Store word from register or immediate memory to memory \\
\end{tabular}

\subsubsection{Arithmetical instructions}

\begin{tabular}{@{}ll@{}}
    ADD         & Add word                                              \\
    %ADDI        & Add immediate word                                    \\
    %ADDIU       & Add immediate unsigned word                           \\
    ADDU        & Add unsigned word                                     \\
    SUB         & Subtract word                                         \\
    %SUBU        & Subtract unsigned word                                \\
    MULTU       & Multiplicate unsigned word                            \\
    BNE         & Branch if non equal                                   \\
    LW          & Load word                                             \\
    SW          & Store word                                            \\
	LWI         & Load immediate word                                   \\
\end{tabular}


\subsubsection{Branch instructions}

\begin{tabular}{@{}ll@{}}
    BNE         & Jump to defined instruction if not equal
\end{tabular}

\subsection{Pipelining}

\end{document}


