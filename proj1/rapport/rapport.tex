\documentclass{report}

\usepackage[T1]{fontenc}
\usepackage[utf8]{inputenc}
\usepackage[english]{babel}

\usepackage{booktabs}
\renewcommand{\arraystretch}{1.2}


\author{Clément Decoodt, Alexis Bauvin, Alexandre Janniaux}
\title{PR1 SE201: execution platforms}

\begin{document}

\maketitle

\section{Two's complement arithmetic}

\subsection{}

\subsection{}

\subsection{}

\subsection{Multiplication and overflow}

\section{Processor design}

\subsection{Instruction set architecture}

We defined our architecture set instruction as the following:

\subsubsection{R-type instructions}

Instruction structure :

\begin{center}
	\begin{tabular}{|l|l|l|l|}
		\hline
		opcode & output register & input register 1 & input register 2 \\
		\hline
		7 bits & 3 bits & 3 bits & 3 bits \\
		\hline
	\end{tabular}
\end{center}

Instruction list :

\begin{center}
	\begin{tabular}{|l|l|l|}
		\hline
		Opcode & Mnemonic & Description \\
		\hline \hline
		\texttt{0000000} & \texttt{NOP} &  Does nothing \\
		\texttt{0000011} & \texttt{ADD} &  Adds two signed numbers \\
		\texttt{0000000} & \texttt{ADDU} & Adds two unsigned numbers \\
		\texttt{0000000} & \texttt{SUB} &  Substracts two signed numbers \\
		\texttt{0000000} & \texttt{SUBU} & Substracts two unsigned numbers \\
		\texttt{0000000} & \texttt{MULU} & Multiplied two unsigned numbers \\
		\hline
	\end{tabular}
\end{center}

\subsubsection{I-type instructions}

\subsubsection{J-type instructions}

\subsubsection{K-type instructions}


\subsection{Pipelining}

\end{document}


